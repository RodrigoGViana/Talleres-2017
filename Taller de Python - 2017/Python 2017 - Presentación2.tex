\documentclass[10pt]{beamer}
\hypersetup{pdfpagemode=FullScreen}
\usepackage[spanish, es-tabla]{babel}  
\selectlanguage{spanish}
\usepackage[utf8]{inputenc}

% ----- Tiene que ver con el tema de la presentación -----------------------------
\usetheme[progressbar=frametitle]{metropolis}
\usepackage{appendixnumberbeamer}
\usepackage{booktabs}
\usepackage[scale=2]{ccicons}
\usepackage{pgfplots}
\usepgfplotslibrary{dateplot}
\usepackage{xspace}
\newcommand{\themename}{\textbf{\textsc{metropolis}}\xspace}
%---------------------------------------------------------------------------------
\usepackage{minted}    % Formato de python
\usepackage{xcolor}
\usepackage{gensymb} % Para generar ciertos símbolos en modo texto.

\title{Taller de Python}
\subtitle{Clase II}
% \date{\today}
\date{}
\author{Comisión de Talleres}
\institute{Centro de Estudiantes Tecnológicos}
%\titlegraphic{\hfill\includegraphics[height=1.5cm]{FondoACET}}

\begin{document}

\maketitle

\begin{frame}{Tabla de contenidos}
  \setbeamertemplate{section in toc}[sections numbered]
  \tableofcontents[hideallsubsections]
\end{frame}

\section{Numpy}
\begin{frame}{¿De qué se trata?}
\textbf{NumPy} es un paquete fundamental para la programación científica que proporciona un objeto tipo \alert{array} para almacenar datos de forma eficiente y una serie de funciones para operar y manipular esos datos. 

Los arrays proporcionados por NumPy son más eficientes que las listas y nos permiten realizar diferentes cálculos.

Para conocer más de lo que ofrece esta librería ir a:

\begin{center}
	\url{http://www.numpy.org/}
\end{center}
\end{frame}

\begin{frame}[fragile]{Importando}
Para utilizar esta librería es necesario importarla.
\begin{minted}[bgcolor=gray!25!white]{python}
import numpy
import numpy as np 
from numpy import * 
\end{minted}
\begin{center}
	¿Da lo mismo ocupar cualquiera de estas opciones?
\end{center}
\end{frame}


\begin{frame}[fragile]{¿Qué nos ofrece?}
Por medio del comando \alert{dir()} es posible saber qué ofrece esta librería.
\begin{minted}[bgcolor=gray!25!white]{python}
dir(numpy) 
\end{minted}
Para saber un poco más acerca de algún elemento en particular, es necesario tener en cuenta la plataforma de trabajo que se esté utilizando.
\begin{minted}[bgcolor=gray!25!white]{python}
numpy.array? # Jupyter notebook
\end{minted}
\end{frame}

\begin{frame}{Ejemplo}
Dado el siguiente conjunto de reacciones en paralelo, hallar expresiones válidas para las velocidades de reacción.

$$
\begin{aligned}
A \xrightarrow{k_{1}}B\\
A \xrightarrow{k_{2}}C\\
A \xrightarrow{k_{3}}D\\
\end{aligned}
$$

\vspace{1em}

\begin{description}
	\item[Concentraciones iniciales]$C_{A0} = 1$ y $C_{B0} = C_{C0} = C_{D0} = 0$ 
	\item[Constantes cinéticas] $k_1 = 4$, $k_2 = 2$ y $ k_3 = 1$
\end{description}
\end{frame}

\begin{frame}[fragile]{Números}
\begin{minted}{python}
'''A las variables a, b y c se le asignan distintos 
valores, como se muestra a continuación.'''

a = 10 
b = 10.0
c = 10 + 10j
	
# Utilice la función type() para saber de qué tipo de
# dato se trata
\end{minted}
\begin{center}
	¿Qué diferencias existen entre los distintos tipos? ¿es posible convertir de un tipo a otro?
\end{center}
\end{frame}

\begin{frame}[fragile]{Strings}
Los \emph{Strings} en Python se identifican como un conjunto de caracteres contiguos encerrados entre comillas.
\begin{minted}[bgcolor=gray!25!white]{python}
'Mi nombre es...' 
\end{minted}
El texto entre comillas es un tipo de dato y es por lo tanto posible operar con él.
\begin{minted}[bgcolor=gray!25!white]{python}
a = 'nombre'
type(a)
len(a)
b = a[0]
2 * a 
\end{minted}

\begin{center}
	¿Qué resultados se obtienen al ejecutar las acciones sugeridas?
\end{center}
\end{frame}

\begin{frame}[fragile]{Listas, tuplas y diccionarios}
	\begin{description}
		\item[Listas] Contienen elementos separados por comas escritos entre corchetes.
		\item[Tuplas] Son semejantes a las listas pero sus elementos están escritos entre paréntesis.
		\item[Diccionarios] Sus elementos consisten en pares \emph{key-value} separados por comas y escritos entre llaves.
	\end{description}

\vspace{1em}

\begin{center}
	¿Qué otras diferencias existen entre estos tipos de datos?
\end{center}
\end{frame}

\begin{frame}[fragile]{Listas, tuplas y diccionarios}
	\begin{minted}{python}
# Lista 
list = ['a', 'b', 3, 5, 7]
	
# Tupla
tuple = ('a', 'b', 3, 5, 7) 

# Diccionario
dict = {1:'Ana', 2:'Bruno', 3:'Carlos'}

list.append(9) # Intente hacer lo mismo con la tupla
list 
list[0]
list[2:3]
	\end{minted}
\end{frame}

\begin{frame}[fragile]{Asignación de variables}
\textbf{Softcoding} es un término que en programación hace referencia al hecho de obtener un valor o función desde una fuente externa. Es lo opuesto de \textbf{hardcoding}, término que hace referencia a programar valores y funciones en el código fuente.

Evitar el hard-coding de valores comúnmente modificados es una buena práctica en programación.

\alert{input()} es una función que permite que el usuario ingrese datos y asignarlos a variables definidas en el código del programa.
	
	\begin{minted}[bgcolor=gray!25!white]{python}
Te = input('Ingrese la temperatura incial en Kelvin: ')
Te = float(Te) 
	\end{minted}
	
	\begin{center}
		¿Por qué necesitamos la segunda línea?
	\end{center}
\end{frame}

\section{Controladores de flujo}
\begin{frame}[fragile]{Condicionales}
   \begin{minted}{python}
number = 23
guess = int(input('Enter an integer : '))

if guess == number:
	print('Congratulations, you guessed it.')
	print('(but you do not win any prizes!)')
elif guess < number:
	print('No, it is a little higher than that')
else:
	print('No, it is a little lower than that')

print('Done')
   \end{minted}
\begin{center}
	Analice el código presentado prestando especial atención a la \emph{identación}.
\end{center}
\end{frame}

\begin{frame}[fragile]{Condicionales}
 \begin{minted}{python}
number = 23
running = True # boolean

while running:
	guess = int(input('Enter an integer : '))
	if guess == number:
		print('Congratulations, you guessed it.')
		running = False
	elif guess < number:
		print('No, it is higher than that.')
	else:
		print('No, it is lower than that.')
print('Done')
\end{minted}
\begin{center}
	¿Qué diferencia(s) hay entre los dos códigos presentados?
\end{center}  
\end{frame}

\begin{frame}[fragile]{Condicionales}
\begin{minted}{python}
for i in range(1, 5):
	print(i)
else:
	print('The for loop is over')
\end{minted}

\begin{center}
	¿Qué diferencia(s) supone ocupar \textbf{for} en lugar de \textbf{while}?  
\end{center}
\end{frame}

\begin{frame}[standout]
	¿Preguntas?
\end{frame}

\section{Repaso}
\begin{frame}{Resumen}
	\begin{table}
		\caption{Operadores ariméticos}
		\begin{tabular}{lr}
			\toprule
			Símbolo&Interpretación\\
			\midrule
		$=$ &Igualdad\\
		$+$ &Suma\\
		$-$ &Resta\\
		$*$ &Multiplicación\\
		$**$&Potencia\\
		$/$ &División\\
		$//$&Parte entera del cociente\\
		$\%$&Resto de la división\\
			\bottomrule
		\end{tabular}
	\end{table}
\end{frame}

\begin{frame}{Resumen}
	\begin{table}
		\caption{Operadores de relación}
		\begin{tabular}{lr}
			\toprule
			Símbolo&Interpretación\\
			\midrule
			$==$ &Igualdad\\
			$!=$&Desigualdad\\
			$<$ &Menor a\\
			$>$ &Mayor a\\
			$<=$ &Menor o igual\\
			$>=$ &Mayor o igual\\
			\bottomrule
		\end{tabular}
	\end{table}
\end{frame}

\begin{frame}{Resumen}
	\begin{table}
		\caption{Controladores de flujo.}
		\begin{tabular}{lr}
			\toprule
			Controlador&Intepretación\\
			\midrule
			if...:&Si se cumple tal cosa, hacer...\\
			elif...:&Si en cambio se cumple otra cosa, hacer...\\ 
			else:&Para todos los demás casos, hacer...\\
			for ... in ...:&Para los elementos en una lista, hacer...\\
			while...:&Mientras se cumpla tal cosa, hacer...\\ 
			\bottomrule
		\end{tabular}
	\end{table}
\end{frame}
\appendix

\begin{frame}{Enlaces de interés}
	\begin{itemize}
		\item \alert{Comunidad Python Argentina}\\ \url{http://www.python.org.ar}
		\item \alert{AeroPython}\\ \url{https://github.com/AeroPython}
		\item \alert{Tutorialspoint}\\ \url{https://www.tutorialspoint.com/index.htm}
		\item \alert{Byte of Python}\\ \url{https://python.swaroopch.com/control_flow.html}
		\item \alert{Python programming}\\ \url{https://pythonprogramming.net}
	\end{itemize}
\end{frame}
\section{Fin de la segunda clase}
\end{document}

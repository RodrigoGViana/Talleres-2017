\documentclass[12pt, oneside, a4paper, spanish]{article}
\usepackage[a4paper,lmargin= 3.5 cm,rmargin=3 cm,top=2.5cm, bottom=2.5cm, headsep=8mm]{geometry}  % Formato de márgenes
\usepackage[spanish, es-tabla]{babel}  
\selectlanguage{spanish}
\usepackage[utf8]{inputenc}
\usepackage{xcolor}
\usepackage{tcolorbox}

\usepackage{booktabs}   % Tablas más bonitas
\usepackage{graphicx} % Insertar imágenes
\usepackage[version=4]{mhchem} % Reacciones y fórmulas químicas.
\usepackage{gensymb} % Para generar ciertos símbolos en modo texto.
\usepackage{amsmath, amsthm, amssymb, mathrsfs} %Para las barras verticales en als derivadas evaluadas en un punto.

\usepackage{fancyhdr}

\pagestyle{fancy}
\fancyhf{}
\lhead{\textcolor{blue!65!black}{Comisión de Talleres}}
\chead{\textbf{Taller de Python}}
\rhead{CET 2017}
\rfoot{}


\begin{document}
\textbf{Funciones y condiciones}
\begin{enumerate}
	\item Defina una función que tome como argumentos dos valores numéricos y devuelva el mayor de ellos. 
	\item Defina una función que tome como argumentos tres valores numéricos y devuelva el mayor de ellos.
	\item Defina una función que tome un carácter y devuelva True si es una vocal, y que de lo contrario devuelva False\footnote{Es importante recordar que Python diferencia entre mayúsculas y minúsculas.}.
	\item Defina una función que devuelva la longitud de su argumento, sea este una lista o cadena de caracteres.
	\item Defina una función que devuelva el mayor valor de una lista de números.
	\item Python cuenta con funciones ya definidas que llevan a cabo estas tareas ¿cuáles son?
\end{enumerate}

\begin{center}
	\textit{El secreto no está en saber todo sino en saber donde hallar las respuestas que necesitamos. Buena suerte.}
\end{center}

\newpage

\textbf{NumPy}

Considere una cañería con las siguientes características, realice las actividades propuestas:

\begin{table}[h]
	\centering
	\begin{tabular}{lll}
		\toprule
		Datos &&\\ 
		\midrule
		Diámetro nominal&3&pulg\\
		Cédula&40&\\
		Diámetro interior&0.0779&m\\
		Rugosidad absoluta&0.05&mm\\
		Longitud&9.60&m\\
		Caudal másico&1&$kg \cdot s^{-1}$\\
		Fluido&Ácido sulfúrico 94\%&\\
		Temperatura&52.78&$\degree$C\\
		Densidad&1801.6&$kg \cdot m^{3}$\\
		Viscosidad&0.01&$Pa \cdot s$ \\
		\bottomrule
	\end{tabular}
\end{table}  

\begin{enumerate}
	\item Diseñe un programa que calcule el \emph{factor de fricción} para el flujo de cualquier fluido a través de tuberías y tubos, por medio de las ecuaciones \eqref{flam} y \eqref{ftur}. El programa debe calcular el número de \emph{Reynolds} y la \emph{rugosidad relativa}. Después, debe tomar decisiones respecto de lo siguiente:
	
	\begin{itemize}
		\item Si $N_{R} < 2000$, emplee la ecuación \eqref{flam}.
		\item Si $2000 < N_{R} < 4000$, el flujo está en el rango de crítico y no es posible calcular un valor confiable de \emph{f}. Genere un mensaje acorde para el usuario del programa.
		\item Si $N_{R} > 4000$, el flujo es turbulento. Emplee la ecuación \eqref{ftur} para calcular \emph{f}.
		\item Imprima $N_{R}, D/\epsilon$ y $f$.
	\end{itemize}
	
	\item Incorpore el programa 1 a otro mejorado para calcular la pérdida de carga para el flujo de cualquier fluido incompresible a través de una tubería de cualquier tamaño. Utilice la ecuación de Darcy, Ec. \eqref{Darcy}  
\end{enumerate}

\begin{tcolorbox}[colback=black!5!white,colframe=white!50!black,title=Ecuaciones a utilizar]
	\begin{align}
	N_R &= \dfrac{vD\rho}{\mu}\\[1em]
	f &= 64/N_{R}\label{flam}\\[1em]
	f &= \dfrac{0.25}{
		\left[log\left(\dfrac{1}{3.7 \cdot (D/\epsilon)}+
		\dfrac{5.74}{N^{0.9}_{R}}\right)\right]^{2}}\label{ftur}\\[1em]
	h_f &= f \dfrac{L}{D} \dfrac{v^2}{2g}\label{Darcy}
	\end{align}
\end{tcolorbox}

\textbf{Matplotlib}

En el ejemplo mostrado se utilizó la función \emph{plot()}, la cual genera segmentos rectos entre cada par de puntos. En las dos siguientes actividades habrá de realizar otros dos tipos sencillos de gráficos, un gráfico de barra y otro de dispersión.

\begin{enumerate}
	\item En un proceso de manufactura de tarjeta electrónicas se quiere investigar la relación entre \emph{X}: rendimiento de pruebas (yield), e \emph{Y}: desperdicio (scrap). Los datos obtenidos son los siguientes:
	
	\begin{table}[h]
		\centering
		\begin{tabular}{llll}
			\toprule
			X & Y & X& Y\\ 
			\midrule
			89.1&2116&89.8&2984\\
			92&1531&91.5&1380\\
			90.1&2717&91&2545\\
			89.5&2004&91.8&1611\\
			88.5&3477&86.7&4814\\
			87.9&3860&89.9&1827\\
			91.7&1947&90.3&3071\\
			91.8&1594&88&4015\\
			91.5&2059&91.1&1975\\
			88.9&2880&90.8&2352\\
			\bottomrule
		\end{tabular}
	\end{table}
¿Qué tipo de relación existe entre las variables? Apóyese en un diagrama de dispersión y el coeficiente de correlación.

\item Con los datos provistos en el ejercicio anterior, genere dos histogramas; uno para cada variable. 
\end{enumerate}
\end{document}
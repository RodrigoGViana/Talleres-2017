\documentclass[12pt, oneside, a4paper, spanish]{article}
\usepackage[a4paper,lmargin= 3.5 cm,rmargin=3 cm,top=2.5cm, bottom=2.5cm, headsep=8mm]{geometry}  % Formato de márgenes
\usepackage[spanish, es-tabla]{babel}  
\selectlanguage{spanish}
\usepackage[utf8]{inputenc}
\usepackage{xcolor}
\usepackage{tcolorbox}

\usepackage{booktabs}   % Tablas más bonitas
\usepackage{graphicx} % Insertar imágenes
\usepackage[version=4]{mhchem} % Reacciones y fórmulas químicas.
\usepackage{gensymb} % Para generar ciertos símbolos en modo texto.
\usepackage{amsmath, amsthm, amssymb, mathrsfs} %Para las barras verticales en als derivadas evaluadas en un punto.

\usepackage{fancyhdr}

\pagestyle{fancy}
\fancyhf{}
\lhead{Comisión de Talleres}
\chead{Taller de Python}
\rhead{CET 2017}
\rfoot{}


\begin{document}
\textbf{Funciones y condiciones}
\begin{enumerate}
	\item Defina una función que tome como argumentos dos valores numéricos y devuelva el mayor de ellos. 
	\item Defina una función que tome como argumentos tres valores numéricos y devuelva el mayor de ellos.
	\item Defina una función que tome un carácter y devuelva True si es una vocal, y que de lo contrario devuelva False\footnote{Es importante recordar que Python diferencia entre mayúsculas y minúsculas.}.
	\item Defina una función que devuelva la longitud de su argumento, sea este una lista o cadena de caracteres.
	\item Defina una función que devuelva el mayor valor de una lista de cualquier longitud.
	\item Python cuenta con funciones ya definidas que llevan a cabo estas tareas ¿cuáles son?
\end{enumerate}

\begin{center}
	\textit{El secreto no está en saber todo sino en saber donde hallar las respuestas que necesitamos. Buena suerte.}
\end{center}

\end{document}
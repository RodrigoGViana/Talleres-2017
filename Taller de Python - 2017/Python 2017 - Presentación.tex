\documentclass[10pt]{beamer}
\hypersetup{pdfpagemode=FullScreen}
\usepackage[spanish, es-tabla]{babel}  
\selectlanguage{spanish}
\usepackage[utf8]{inputenc}

% ----- Tiene que ver con el tema de la presentación -----------------------------
\usetheme[progressbar=frametitle]{metropolis}
\usepackage{appendixnumberbeamer}
\usepackage{booktabs}
\usepackage[scale=2]{ccicons}
\usepackage{pgfplots}
\usepgfplotslibrary{dateplot}
\usepackage{xspace}
\newcommand{\themename}{\textbf{\textsc{metropolis}}\xspace}
%---------------------------------------------------------------------------------
\usepackage{minted}    % Formato de python
\usepackage{xcolor}
\usepackage{gensymb} % Para generar ciertos símbolos en modo texto.

\title{Taller de Python}
\subtitle{Clase I}
% \date{\today}
\date{}
\author{Comisión de Talleres}
\institute{Centro de Estudiantes Tecnológicos}
%\titlegraphic{\hfill\includegraphics[height=1.5cm]{FondoACET}}

\begin{document}

\maketitle

\begin{frame}{Tabla de contenidos}
  \setbeamertemplate{section in toc}[sections numbered]
  \tableofcontents[hideallsubsections]
\end{frame}

\section{Introducción}
\begin{frame}[fragile]{Descarga e Instalación}
Para llevar a cabo las actividades propuestas, recomendamos descargar \alert{Anaconda} de Continuun siguiendo el enlace que ofrecemos a continuación.
\vspace{1em}
\begin{center}
	\url{https://www.continuum.io/downloads}
\end{center}
\end{frame}

\begin{frame}{¿Qué es Python?}
Python es un \alert{lenguaje de programación} con las siguientes características:

\begin{itemize}[<+(1)- | alert@+>]
	\item \texttt{Es un lenguaje de propósito general},
	\item \texttt{interpretado},
	\item \texttt{multiparadigma},
	\item \texttt{fuertemente tipado}.
\end{itemize}

\begin{center}
	\only<6>{¡También es muy sencillo!}
\end{center}
\end{frame}

\section{Funciones}
\begin{frame}[fragile]{Funciones}
\alert{Una función es un fragmento de código con un nombre asociado, reutilizable y que permite llevar a cabo una tarea en particular.}

\vspace{1em}

El interprete de Python tiene un número de funciones y tipos siempre disponibles. 
	
\begin{center}
\url{https://docs.python.org/3/library/functions.html}
\end{center}
\end{frame}

\begin{frame}[fragile]{Funciones}
Para definir nuestras propias funciones, usamos el comando \textbf{def}.

\begin{minted}[bgcolor=gray!25!white]{python}
def F(x,y):
	return x + y**2
	
def G():
	print('Hello world!')
\end{minted}

\begin{itemize}
		\item La línea que contiene al comando def termina siempre con ':'.
		\item Todos las sentencias que formen parte de la definición de la función deben estar correctamente identadas.
\end{itemize}
\begin{center}
		¿Qué diferencia existe entre \textbf{return} y \textbf{print}?
\end{center}

\end{frame}

\section{Variables}
\begin{frame}[fragile]{Tipos de datos}
	Python tiene cinco \alert{tipos de datos estándares}:
	\begin{enumerate}
		\item Números
		\item \emph{Strings}
		\item Listas
		\item Tuplas
		\item Diccionarios
	\end{enumerate}
\end{frame}

\begin{frame}[fragile]{Números}
\begin{minted}{python}
'''A las variables a, b y c se le asignan distintos 
valores, como se muestra a continuación.'''

a = 10 
b = 10.0
c = 10 + 10j
	
# Utilice la función type() para saber de qué tipo de
# dato se trata
\end{minted}
\begin{center}
	¿Qué diferencias existen entre los distintos tipos? ¿es posible convertir de un tipo a otro?
\end{center}
\end{frame}

\begin{frame}[fragile]{Strings}
Los \emph{Strings} en Python se identifican como un conjunto de caracteres contiguos encerrados entre comillas.
\begin{minted}[bgcolor=gray!25!white]{python}
'Mi nombre es...' 
\end{minted}
El texto entre comillas es un tipo de dato y es por lo tanto posible operar con él.
\begin{minted}[bgcolor=gray!25!white]{python}
a = 'nombre'
type(a)
len(a)
b = a[0]
2 * a 
\end{minted}

\begin{center}
	¿Qué resultados se obtienen al ejecutar las acciones sugeridas?
\end{center}
\end{frame}

\begin{frame}[fragile]{Listas, tuplas y diccionarios}
	\begin{description}
		\item[Listas] Contienen elementos separados por comas escritos entre corchetes.
		\item[Tuplas] Son semejantes a las listas pero sus elementos están escritos entre paréntesis.
		\item[Diccionarios] Sus elementos consisten en pares \emph{key-value} separados por comas y escritos entre llaves.
	\end{description}

\vspace{1em}

\begin{center}
	¿Qué otras diferencias existen entre estos tipos de datos?
\end{center}
\end{frame}

\begin{frame}[fragile]{Listas, tuplas y diccionarios}
	\begin{minted}{python}
# Lista 
list = ['a', 'b', 3, 5, 7]
	
# Tupla
tuple = ('a', 'b', 3, 5, 7) 

# Diccionario
dict = {1:'Ana', 2:'Bruno', 3:'Carlos'}

list.append(9) # Intente hacer lo mismo con la tupla
list 

list[0]
list[2:3]
	\end{minted}
\end{frame}

\begin{frame}[fragile]{Asignación de variables}
\textbf{Softcoding} es un término que en programación hace referencia al hecho de obtener un valor o función desde una fuente externa. Es lo opuesto de \textbf{hardcoding}, término que hace referencia a programar valores y funciones en el código fuente.

Evitar el hard-coding de valores comúnmente modificados es una buena práctica en programación.

\alert{input()} es una función que permite que el usuario ingrese datos y asignarlos a variables definidas en el código del programa.
	
	\begin{minted}[bgcolor=gray!25!white]{python}
Te = input('Ingrese la temperatura incial en Kelvin: ')
Te = float(Te) 
	\end{minted}
	
	\begin{center}
		¿Por qué necesitamos la segunda línea?
	\end{center}
\end{frame}

\section{Controladores de flujo}
\begin{frame}[fragile]{Condicionales}
   \begin{minted}{python}
number = 23
guess = int(input('Enter an integer : '))

if guess == number:
	print('Congratulations, you guessed it.')
	print('(but you do not win any prizes!)')
elif guess < number:
	print('No, it is a little higher than that')
else:
	print('No, it is a little lower than that')

print('Done')
   \end{minted}
\begin{center}
	Analice el código presentado prestando especial atención a la \emph{identación}.
\end{center}
\end{frame}

\begin{frame}[fragile]{Condicionales}
 \begin{minted}{python}
number = 23
running = True # boolean

while running:
	guess = int(input('Enter an integer : '))
	if guess == number:
		print('Congratulations, you guessed it.')
		running = False
	elif guess < number:
		print('No, it is higher than that.')
	else:
		print('No, it is lower than that.')
print('Done')
\end{minted}
\begin{center}
	¿Qué diferencia(s) hay entre los dos códigos presentados?
\end{center}  
\end{frame}

\begin{frame}[fragile]{Condicionales}
\begin{minted}{python}
for i in range(1, 5):
	print(i)
else:
	print('The for loop is over')
\end{minted}

\begin{center}
	¿Qué diferencia(s) supone ocupar \textbf{for} en lugar de \textbf{while}?  
\end{center}
\end{frame}

\begin{frame}[standout]
	¿Preguntas?
\end{frame}

\section{Repaso}
\begin{frame}{Resumen}
	\begin{table}
		\caption{Operadores ariméticos}
		\begin{tabular}{lr}
			\toprule
			Símbolo&Interpretación\\
			\midrule
		$=$ &Igualdad\\
		$+$ &Suma\\
		$-$ &Resta\\
		$*$ &Multiplicación\\
		$**$&Potencia\\
		$/$ &División\\
		$//$&Parte entera del cociente\\
		$\%$&Resto de la división\\
			\bottomrule
		\end{tabular}
	\end{table}
\end{frame}

\begin{frame}{Resumen}
	\begin{table}
		\caption{Operadores de relación}
		\begin{tabular}{lr}
			\toprule
			Símbolo&Interpretación\\
			\midrule
			$==$ &Igualdad\\
			$!=$&Desigualdad\\
			$<$ &Menor a\\
			$>$ &Mayor a\\
			$<=$ &Menor o igual\\
			$>=$ &Mayor o igual\\
			\bottomrule
		\end{tabular}
	\end{table}
\end{frame}

\begin{frame}{Resumen}
	\begin{table}
		\caption{Operadores lógicos}
		\begin{tabular}{lr}
			\toprule
			Símbolo&Interpretación\\
			\midrule
			and & y\\
			or  & o\\
			not & no\\
			\bottomrule
		\end{tabular}
	\end{table}
\end{frame}

\begin{frame}{Resumen}
	\begin{table}
		\caption{Controladores de flujo.}
		\begin{tabular}{lr}
			\toprule
			Controlador&Intepretación\\
			\midrule
			if...:&Si se cumple tal cosa, hacer...\\
			elif...:&Si en cambio se cumple otra cosa, hacer...\\ 
			else:&Para todos los demás casos, hacer...\\
			for ... in ...:&Para los elementos en una lista, hacer...\\
			while...:&Mientras se cumpla tal cosa, hacer...\\ 
			\bottomrule
		\end{tabular}
	\end{table}
\end{frame}
\appendix

\begin{frame}{Enlaces de interés}
	\begin{itemize}
		\item \alert{Comunidad Python Argentina}\\ \url{http://www.python.org.ar}
		\item \alert{AeroPython}\\ \url{https://github.com/AeroPython}
		\item \alert{Tutorialspoint}\\ \url{https://www.tutorialspoint.com/index.htm}
		\item \alert{Byte of Python}\\ \url{https://python.swaroopch.com/control_flow.html}
		\item \alert{Python programming}\\ \url{https://pythonprogramming.net}
	\end{itemize}
\end{frame}
\section{Fin de la primera clase}
\end{document}
